\documentclass{article}
\usepackage[utf8]{inputenc}
\usepackage[inline]{enumitem}
\usepackage{tcolorbox}
\usepackage{hyperref}
% \usepackage[margin=1in]{geometry}

\title{How to write a good project plan:\\%
       Helpful tips for starting up MSc and BSc projects}
\author{Alexander Neergaard Zahid, Postdoc, DTU Compute\\%
        Morten Mørup, Professor, DTU Compute}
\date{February, 2023}

\begin{document}

\maketitle

\begin{abstract}
    Writing and submitting a project plan is the first step, when beginning your MSc/BSc project. 
    However, there is not a lot of official guidance provided by DTU about how to specifically set up and write a good project plan.
    In this document, we will try to provide tips and tricks for writing a plan, including which elements to include, how to set up and formulate research questions, and how to scope your time schedule.
\end{abstract}

\section{Why having a project plan is a good idea}
The project plan is the first major document you will be writing during your thesis work, and it is to be submitted approximately one month after starting. 
Although it is not as long or detailed as your thesis will be, it is still a good idea to take some time into making sure the project plan is thorough.
Apart from being an official requirement by DTU, you should think of the project plan as a way to align the expectations of yourself and your supervisor(s), as well as any potential collaborators.
It is also an excellent opportunity for you to really dive in to the subject and \emph{condense} the subject matter into formalized investigation points or \emph{research questions}.

A good project plan should be short (2-3 pages), concise, and could ideally serve as a first draft of the \emph{Introduction} in your thesis.

\section{What to include in your project plan}
Formally, the project plan should include (at least) these sections:
\begin{enumerate}
    \item \textbf{Introduction},
    \item \textbf{Research questions},
    \item \textbf{MSc/BSc learning objectives},
    \item \textbf{Time schedule}, and
    \item \textbf{References}
\end{enumerate}
The specific content of these section is detailed below.
It should be noted that this is a \textit{sufficient}, not \textit{necessary}, list.

\subsection{Introduction}
This section could also be called \emph{Project description} or \emph{Motivating background}, and it is where the majority of the text body will be.
While this may vary across disciplines and departments, the contents in this section could ideally be contained in three paragraphs.

The first will introduce and elaborate on the underlying problem that you are trying to solve.
Think of it as a way to introduce an unfamiliar reader to the topic and make them understand why this is interesting or important.

The second paragraph will describe at a high level the ways that this problem has been investigated previously in related work by others.
This should not be too technical, so don't go into details about model architectures, hyperparameters etc. 
Try to find a balance, such that a new reader will get the feel for the state of the art, and such that you yourself can remember related work.

The third paragraph should contain a brief description of what your project will do to solve the problem introduced in the first paragraph.
Think of ways to describe your method, algorithm, model or similar, in a way that emphasizes how it is different and/or novel.


\subsection{Research questions}

\subsection{MSc/BSc learning objectives}

The following items are the specific BSc learning objectives, which has to be included in the project plan.
\begin{figure}
    \caption{Specific learning objectives for BSc theses.}
    \begin{tcolorbox}[colframe=white]
    The BSc graduate from DTU
    \begin{itemize}
        \item can work independently and is able to structure a major project, including meeting deadlines and organizing and planning the project work
        \item can summarize and interpret technical information and is fully familiar with technical problem solving through project work
        \item is able to work with all project phases, including the preparation of proposals, solutions, and documentation
        \item is able to independently acquire new knowledge and adopt a critical approach to the acquired knowledge and carry out relevant and critical information searches, and on this basis find the right methods to shed light on the problem in question
        \item is able to communicate technical information, theory, and results in written visual/graphic, and oral form.
    \end{itemize}
    \end{tcolorbox}
\end{figure}
\begin{figure}
    \begin{tcolorbox}[colframe=white]
    A graduate of the MSc programme from DTU:
    \begin{itemize}
        \item can identify and reflect on technical scientific issues and understand the interaction between the various components that make up an issue
        \item can, based on a clear academic profile, apply elements of current research at an international level to develop ideas and solve problems 
        \item masters technical scientific methodologies, theories, and tools, and can take a holistic view of and delimit a complex, open issue, put it into a broader academic and societal perspective, and, on this basis, propose a variety of possible actions while considering sustainability
        \item can develop relevant models, systems, technologies, and processes aimed at solving technological problems
        \item can communicate and mediate research-based knowledge both orally and in writing
        \item is familiar with and can seek out leading international research within their specialist area
        \item can work independently and reflect on own learning, academic development, and specialization
        \item masters technical problem-solving at a high level through cross-disciplinary teamwork, and can work with and manage all phases of a project--including preparation of timetables, design, solution, and documentation
    \end{itemize}
    \end{tcolorbox}
    \caption{Specific learning objectives for MSc theses.}
\end{figure}


\subsection{Time schedule}
A larger project will typically contain specific phases, during which certain tasks are to be completed.
For instance, the first couple of weeks will typically be spent on familiarizing yourself with relevant literature and reading tons of papers, while the last couple of weeks will be spent entirely focused on writing your thesis.
In between these periods might be weeks dedicated to data collection, exploratory analysis, model development, testing and validation, and result analysis, and these typically follow an ordering of sorts.
This section should be used to showcase the project structure from initiation to final submission.

Typically, this is realized using a Gantt chart\footnote{\url{https://en.wikipedia.org/wiki/Gantt_chart}}, which is a convenient way to structure a project into specific tasks spanning across time slots, and deadlines.

\subsection{References}
It is important to back up the descriptive text in \emph{Introduction} with references to literature, and this section will contain an itemized list of any cited papers, documents, books, and similar. 


\end{document}
